\chapter{Budowa śródowiska}

Praktyczna realizacja zaproponowanego rozwiązania składa się z dwóch części: platformy fotogrametrycznej i zestawu skryptów.

\section{Platforma AliceVision}

\textbf{AliceVision} --- framework fotogrametryczny z otwartym kodem źródłomym, udostępniany pod licensją \textbf{MPL2}.
Platforma została stworzona aby umożliwić odtwarzanie sceny fotogrametrycznej na podstawie zbioru zdjęć lub sekwencji klatek wideo zawierających scene.
Do wybranych możliwości platformy należą:
\begin{itemize}
   \item Wyznaczanie punktów charakterystycznych --- cech sceny
   \item Parowanie zdjęć --- para jest tworzony gdy zdjęcia zawierają ten sam obiekt
   \item Zestawienie cech zdjęć
   \item Odtwarzanie struktury cech w przestrzeni 3D
   \item Lokalizacja względna i bezwzględna kamery
   \item Kalibracja kamery
   \item Utworzenie mapy głębokości sceny
   \item Odworzenie powierzchni obiektów sceny
   \item Teksturowanie obiektów sceny
\end{itemize}

Aby wykorzystać framework należy pobrać kod źródłowy projektu i zbudować go w swoim śródowisku.
Istnieje również możliwość pobrania już gotowego projektu w postaci narżedzia \textbf{Meshroom}, program ten jest swojego rodzaju powłoką graficzną dla \textbf{AliceVision} i pozwala w łatwy sposób zapoznać się z możliwościami i sposobem działania platformy.
Pliki frameworku znajdują się w katalogu głównym programu.

Zbudowany framework przedstawia sobą zestaw plików binarnych - pliki wykonywalne i biblioteki dynamiczne.
Każdy plik wykonywalny przedstawia sobą osobne narzędzie.

Wykorzystanie i komunikacja z \textbf{AliceVision} odbywa się w sposób uruchomienia odpowiedniego pliku wykonywalnego w powłoce systemu (np linii poleceń) z odpowiednim zestawem parametrów.
Wszystkie operacje odczytu/zapisu odbywają się na plikach, ścieżka których jest określana w parametrach.
Na potrzeby pracy wykorzystano następujące narzędzia:

\begin{enumerate}
   \item \textbf{CameraInit}: inicjacja danych, odczyt parametrów kamery,
   \item \textbf{FeatureExtraction}: wyszukiwanie cech,
   \item \textbf{ImageMatching}: parowanie zdjęć,
   \item \textbf{FeatureMatching}: parowanie cech,
   \item \textbf{StructureFromMotion}: odtwarzanie struktury cech w przestrzeni, lokalizacja kamery.
\end{enumerate}

Narzędzia są opisane w kolejności odpowiedniej do kolejności uruchomienia, taka kolejność zapewnia przetwarzanie potokowe danych.
Każde narzędzie jest opisane w poszczególnych rozdziałach pod względem danych wejścia/wyjścia i zastosowanych parametrach w przetwarzaniu danych.

\subsection{CameraInit}
\subsection{FeatureExtraction}
\subsection{ImageMatching}
\subsection{FeatureMatching}
\subsection{StructureFromMotion}

Zaletą \textbf{AliceVision} można nazwać wieloplatformowość, framework można zbudować dla takich systemów operacyjnych, jak Windows, Linux lub OSX.
Do zalet platformy AliceVision można również odnieść ten fakt, iż jej narzędzia zaimplementowane z wykorzystaniem takich technologii, jak na przykład \textbf{Mosek}, stosowanie których pozwala zoptymalizować i przyspieszyć wykonanie części algorytmów zwłaszcza algorytmy liniowe.
Stosowanie \textbf{OpenMP} pozwala na obliczenia wielowonkowe, a \textbf{CUDA} czy \textbf{OpenCL} umożliwiają przyrost poprzez wykonanie prostych operacji algebraicznych na procesorach graficznych.

Framework powstał i jest rozwijany jako projekt wspólny przez takie centra naukowe, jak:
\begin{itemize}
   \item Czech Technical University (CTU) in Prague, Czech Republic
   \item Institut National Polytechnique de Toulouse (Toulouse INP), France
   \item Mikros Image, Post-Production Company in Paris, France
   \item Simula Research Laboratory AS in Oslo, Norway
   \item Quine in Oslo, Norway
   \item Wspierany przez European Union’s Horizon 2020
\end{itemize}

\section{Implementacja śródowiska}

Dla realizacji zaproponowanego algorytmu i wyświetlania wyników został napisany zestaw skryptów w języku \textbf{Python} wersji 3.8.1 pozwalające na:
\begin{itemize}
   \item pojedyńcze przetwarzanie zestawu zdjęć sceny w celu odtwarzania sceny i lokalizacji kamery,
   \item wyświetlanie odtwarzonej sceny w postaci punktów kluczowych i położenia kamer,
   \item wyświetlanie wybranych punktów kluczowych w przestrzeni i odpowiednich zdjęciach,
   \item przeprowadzanie testów wydajności dla zbioru róźnych zestawów danych i parametrów przetwarzania,
\end{itemize}

\subsection{Pipeline}
\subsection{Rysowanie}
\subsection{Pomiar czasu}
