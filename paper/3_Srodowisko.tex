\chapter{Budowa śródowiska}

Dla praktycznego sprawdzenia zostało napisane sródowisko składające się z narzędzi:

\section{Platforma AliceVision}

\textbf{AliceVision} --- framework fotogrametryczny stworzony był aby umożliwić odtworzenie \textbf{sceny fotogrametrycznej} na podstawie zbioru zdjęć lub sekwencji klatek wideo zawierających scene.

Wybrane możliwości platformy umożliwiają:
\begin{itemize}
\item Wyznaczanie punktów charakterystycznych --- cech sceny
\item Parowanie zdjęć --- para jest tworzony gdy zdjęcia zawierają ten sam obiekt
\item Zestawienie cech zdjęć
\item Odtworzenie struktury cech w przestrzeni 3D
\item Lokalizacja względna i bezwzględna kamery
\item Kalibracja kamery
\item Utworzenie mapy głębokości sceny
\item Odworzenie powierzchni obiektów sceny
\item Teksturowanie obiektów sceny
\end{itemize}

\subsection{CameraInit}
\subsection{FeatureExtraction}
Najwiej
\subsection{ImageMatching}
\subsection{FeatureMatching}
\subsection{StructureFromMotion}
\section{Implementacja śródowiska}
\begin{itemize}
\item przetwarzania danych wejściowych w celu zestawienia zdjęć i lokalizacji kamery
\item wyświetlania wyników fotogrametrycznych i położenia kamery w przestrzeni
\item wyświetlania elementów kluczowych w przestrzeni i odpowiednich obrazach
\item przeprowadzania testów wydajności dla róźnych zestawów danych i parametrów przetwarzania
\end{itemize}

\subsection{Pipeline}
\subsection{Rysowanie}
\subsection{Pomiar czasu}
