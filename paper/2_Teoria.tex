\graphicspath{ {./img/2_Theory/} }


\chapter{Wprowadzenie teoretyczne}
W tym rozdziale proces jest opisany na przykładzie działania AliceVision.
\section{Fotogrametria}
Fotogrametria - dziedzina nauki, sztuki i technologia mająca na celu wyznaczanie informacji i odtworzanie kształtów jednego lub więcej obiektu fizycznego na podstawie zdjęć obiektu.
Fotogrametria można podizelić na dwa rodzaje: płaską i przestrzenną. Fotogrametria płaska wymaga co najmniej jednego zdjęcia i jej wynikiem jest charakterystyka obiektu w płaszczyźnie. Fotogrametria przestrzenna wymaga dwóch lub więcej zdjęć, jest używana do odtwarzania obiektów w przestrzeni trójwymiarowej.

W tej pracy stosowana jest wyłącznie fotogrametria przestrzenna.

Pozwala wyznaczyć takie charakterystyki jak:
\begin{itemize}
   \item krszałt
   \item rozmiar
   \item położenie wzajemne w przestrzeni
   \item color i teksturę
\end{itemize}

Dodatkowo technologia fotogrametrii pozwala na wyznaczenie położenia kamery(obserwatora) w chwili robienia zdjęcia fotogrametrycznego.
Położenie kamery w przestrzeni trójwymiarowej jest wyznaczane na podstawie bazy sceny i wybranego zdjęcia zawierającego całość lub część sceny.

\subsection{Wyznaczanie punktów kluczowych}
Obraz z zestawu zdjęć fotogrametrycznych posiada cechy kluczowe które mogą być na tyle charakterystyczne, że można z wysokim prawdopodobieństwem je znaleźć w bazie cech innego zdjęcia zawierającego ten sam obiekt.

Jednym ze sposobów wyznaczania i zapisu takich elementów jest stosowanie opatentowanych algorytmów i deskryptorów \textbf{SIFT} (Scale Invariant Feature Transform). Ta transformata jest powszechnie stosowana w przetwarzaniu i rozpoznawaniu obrazów do wykrywania punktów charakterystycznych.

Aby zminimalizować zapotrzebowanie zasobów na wyznaczanie elementów kluczowych obraz jest poddawany szeregu wstępnych zabiegów optymalizacji i filtracji. Taki proces wyznaczania cech można przedstawić w taki szereg kroków:
\begin{enumerate}
   \item \textbf{Wykrywanie ekstremów dla róźnych skal}:
      wyszukuje miejsca na obrazie które mogą zawierać elementy kluczowe za pomocą \textbf{DoG}. Funkcja DoG (Difference of Gaussians - Róźnica filtrów Gaussa) jest przybliżeniem \textbf{LoG} (Laplasjan filtru Gaussa) - funkcją filtrują która definiuje macierz konwolucji dla filtracji obrazu. Filtracja obrazu wykrywa krewędzie nie zależnie od ich skali i obrotu, w taki sposób można odrzucić obszary obrazu, dalsza analiza których jest zbędna.

      \begin{figure}[h!]
         \centering
         \includegraphics[width=5cm]{DoG_orgin.png}
         {\Large $  \xrightarrow{DoG}  $} \vspace{2cm}
         \includegraphics[width=5cm]{DoG_F_white.png}
      \end{figure}

      [DoG before-after image]

   \item \textbf{Lokalizacja punktów kluczowych}

   \item \textbf{Przydział obrotu}

   \item \textbf{Lokalizacja punktów kluczowych}

   \item \textbf{Lokalizacja punktów kluczowych}
\end{enumerate}



\subsection{Parowanie obrazów}
Aby przyspieszyć wykonanie następnego kroku - \textbf{zestawienie punktów kluczowych}, wprowadzono zestawienie
\subsection{Zestawienie punktów kluczowych}
\subsection{Wyznaczanie kształtów sceny}
\section{Lokalizacja kamery}




SK:
Fotogrametria

Punkt charakterystyczny
Element kluczowy
Cechy charakterystyczne
SIFT (Scale Invariant Feature Transform - Transformata )
Deskryptor
