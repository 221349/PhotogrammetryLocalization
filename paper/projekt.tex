%Przykładowy plik ułatwiający złożenie projektu dyplomowego inżynierskiego.
%UWAGA: Generowany napis na stronie tytułowej o treści PROJEKT DYPLOMOWY INŻYNIERSKI został zaproponowany przeze mnie i nie jest, póki co, potwierdzony przez władze wydziału. Przed ostatecznym oddaniem tak złożonej pracy należy upewnić się jaka powinna być treść tego napisu. W momencie gdy uzyskam informację na temat treści tego napisu, dokonam niezbędnych zmian w źródłach.

\documentclass[eng,printmode]{mgr}
%opcje klasy dokumentu mgr.cls zostały opisane w dołączonej instrukcji


\usepackage{polski}
\usepackage[utf8x]{inputenc}

%pakiety do grafiki
\usepackage{graphicx}
\usepackage{subfigure}
\usepackage{psfrag}

%pakiety dodające dużo dodatkowych poleceń matematycznych
\usepackage{amsmath}
\usepackage{amsfonts}

%pakiety wspomagające i poprawiające składanie tabel
\usepackage{supertabular}
\usepackage{array}
\usepackage{tabularx}
\usepackage{hhline}

%pakiet wypisujący na marginesie etykiety równań i rysunków zdefiniowanych przez \label{}, chcąc wygenerować finalną wersję dokumentu wystarczy usunąć poniższą linię
\usepackage{showlabels}

%definicje własnych poleceń
\newcommand{\R}{I\!\!R} %symbol liczb rzeczywistych, działa tylko w trybie matematycznym
\newtheorem{theorem}{Twierdzenie}[section] %nowe otoczenie do składania twierdzeń

%dane do złożenia strony tytułowej
\title{Zastosowanie algorytmów fotogrametrii dla estymacji położenia obserwatora w przestrzeni trójwymiarowej}
\engtitle{A photogrammetric approach for observer's position estimation in 3D}
\author{Lev Sergeyev}
\supervisor{Dr hab. inż. Przemysław Śliwiński, prof. PWr}
\guardian{Prof. dr hab. inż. Czesław Smutnicki, prof. PWr}

\date{2020} %standardowo u dołu strony tytułowej umieszczany jest bieżący rok, to polecenie pozwala wstawić dowolny rok

\field{Automatyka i Robotyka (AIR)}
\specialisation{Systemy informatyczne w automatyce (ASI)}

%tutaj zaczyna się właściwa treść dokumentu
\begin{document}
%\bibliographystyle{plabbrv} %tylko gdy używamy BibTeXa, ustawia polski styl bibliografii

\maketitle %polecenie generujące stronę tytułową
%\dedication{6cm}{To jest przykładowa treść opcjonalnej dedykacji, należy ją zmienić lub usunąć w całości polecenie \texttt{$\backslash$dedication}}

\tableofcontents %spis treści


%%%%%%%%%%%%%%%%%%%%%%%%%%
%% DOCUMENT:            %%
%%%%%%%%%%%%%%%%%%%%%%%%%%

%poniżej znajduje się przykładowa treść dalszej części dokumentu, zainteresowanych zachęcam do rozszyfrowania frazy "Lorem ipsum" :)
\chapter{Wstęp}
\section{Problem}
\section{Cel pracy}

\chapter{Wprowadzenie teoretyczne}
\section{Fotogrammetria}
\subsection{krok 1}
\subsection{krok ..}
\subsection{krok 5}
\section{Lokalizacja kamery}

\chapter{Budowa śródowiska}
\section{Framework AliceVision}
\subsection{CameraInit}
\subsection{FeatureExtraction}
\subsection{ImageMatching}
\subsection{FeatureMatching}
\subsection{StructureFromMotion}
\section{Implementacja śródowiska}
\subsection{Pipeline}
\subsection{Rysowanie}
\subsection{Pomiar czasu}

\chapter{Przebieg badań}
\section{Wpływ danych na wynik i czas obliczenia}
\subsection{Jakość obrazów}
\subsection{Ilość obrazów}
\subsection{Zawartość obrazów}
\section{Wpływ parametrów na wynik i czas obliczenia}
\subsection{Jakość elementów kluczowych}
\subsection{Ilość elementów kluczowych}
\section{Eksperyment z pomiarem odległości}

\chapter{Wnioski}

\begin{itemize}
\item list item
\end{itemize}


\appendix
\chapter{Dodatek}

\addcontentsline{toc}{chapter}{\bibname} %utworzenie w spisie treści pozycji Literatura
\bibliography{bibliografia} % wstawia bibliografię korzystając z pliku bibliografia.bib - dotyczy BibTeXa, jeżeli nie korzystamy z BibTeXa należy użyć otoczenia thebibliography

%opcjonalnie może się tu pojawić spis rysunków i tabel
% \listoffigures
% \listoftables
\end{document}
